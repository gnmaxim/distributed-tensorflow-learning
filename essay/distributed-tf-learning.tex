\documentclass[DIV=calc, paper=a4, fontsize=11pt]{scrartcl}	 % A4 paper and 11pt font size
%\documentclass[DIV=calc, paper=a4, fontsize=11pt, twocolumn]{scrartcl}	 % A4 paper and 11pt font size

\usepackage{multicol}
\usepackage{lipsum}
\usepackage[italian]{babel}
\usepackage[protrusion=true,expansion=true]{microtype}
\usepackage{amsmath,amsfonts,amsthm}
\usepackage[svgnames]{xcolor}
\usepackage[hang, small,labelfont=bf,up,textfont=it,up]{caption}
\usepackage{booktabs}
\usepackage{fix-cm}
\usepackage[margin=1in]{geometry}
\usepackage[utf8]{inputenc}
\setlength{\columnsep}{.6cm}

\usepackage{sectsty}
\allsectionsfont{\usefont{OT1}{phv}{b}{n5}}
\usepackage{fancyhdr}
\pagestyle{fancy}
\usepackage{lastpage}

% Headers - all currently empty
\lhead{}
\chead{}
\rhead{}

% Footers
\lfoot{}
\cfoot{}
\rfoot{\footnotesize Page \thepage\ of \pageref{LastPage}} % "Page 1 of 2"

\renewcommand{\headrulewidth}{.0pt} % No header rule
\renewcommand{\footrulewidth}{.4pt} % Thin footer rule

\usepackage{lettrine} % Package to accentuate the first letter of the text
\newcommand{\initial}[1]{ % Defines the command and style for the first letter
\lettrine[lines=3,lhang=0.3,nindent=0em]{
\color{DarkGoldenrod}
{\textsf{#1}}}{}}

\usepackage{titling} % Allows custom title configuration

\newcommand{\HorRule}{\color{DarkGoldenrod} \rule{\linewidth}{1pt}} % Defines the gold horizontal rule around the title

\pretitle{\vspace{-30pt} \begin{flushleft} \HorRule \fontsize{30}{30} \usefont{OT1}{phv}{b}{n} \color{DarkRed} \selectfont} % Horizontal rule before the title

\title{TensorFlow e Prove di Apprendimento Distribuito} % Your article title

\posttitle{\par\end{flushleft}\vskip 2em} % Whitespace under the title

\preauthor{\begin{flushleft}\large \lineskip 0.4em \usefont{OT1}{phv}{b}{sl} \color{DarkRed}} % Author font configuration

\author{Maxim Gaina e Bartolomeo Lombardi} % Your name

\postauthor{\footnotesize \usefont{OT1}{phv}{m}{sl} \color{Black} % Configuration for the institution name
\\ Lavoro di progetto per Sistemi Peer-to-Peer, Università degli Studi di Bologna % Your institution

\par\end{flushleft}\HorRule} % Horizontal rule after the title

\date{} % Add a date here if you would like one to appear underneath the title block


\begin{document}
	\maketitle
	\thispagestyle{fancy}
	% The first character should be within \initial{}
	\initial{Q}\textbf{uesto lavoro mira ad apprendere e riassumere le basi di TensorFlow \cite{tf}, il quale, come da definizione, è un'interfaccia per esprimere algoritmi nell'ambito di Machine Learning con anche la possibilità di implementarli. Successivamente l'intenzione sarà quella di esplorare le primitive offerte in ambito del calcolo distribuito e, dato un problema facilmente risolvibile tramite reti neurali, fare delle prove pratiche per ottenere una soluzione soddisfacente usando un sistema di macchine connesse.}
	
	\begin{multicols}{2}
		\tableofcontents
		\section{Concetti base di TensorFlow}
			TensorFlow nasce dalla necessità di avere un sistema con le giuste proprietà e requisiti per \textit{allenare} e usare reti neurali in ambienti distribuiti su larga scala e non. La computazione di un programma scritto in TensorFlow è eseguibile su piattaforme multiple ed eterogenee con poco o senza alcun cambiamento. Per ogni piattaforma presa in considerazione, è previsto lo sfruttamento dei diversi device a sua disposizione con capacità di calcolo, come CPU (processori centrali) e GPU (acceleratori di grafica). Le computazioni vengono espresse da flussi di dati che scorrono all'interno di un grafo che possiede uno stato. Ulteriori obiettivi, sono quelli di fornire un linguaggio \textit{flessibile}, che permette la rapida implementazione di diversi modelli; un linguaggio il più possibile \textit{performante} nonostante la flessibilità; e forme di parallelismo con requisiti più o meno forti, per passare con facilità da ambienti isolati ad ambienti distribuiti.		
		\section{APPUNTI}
		L'articolo del 2015 cercherà di essere aggiornato con qualcosa che c'è sul sito.
			\begin{itemize}
				\item Paradigma e Concetti Base
				\begin{itemize}
					\item Operazioni e Kernel
					\item Sessioni
					\item Variabili
					\item[my] Placeholders?
				\end{itemize}
				\item Implementazione
				\begin{itemize}
					\item Devices
					\item Tensori
					\item Esecuzione su singolo e multiplo device
					\begin{itemize}
						\item dove collocare nodi
						\item come comunicare fra nodi
					\end{itemize}
					\item Esecuzione distribuita, tolleranza fault
				\end{itemize}
				
				
				\begin{itemize}
					\item
				\end{itemize}
			\end{itemize}
		
		\begin{description}
		\item[First] This is the first item
		\item[Last] This is the last item
		\end{description}
		
		\section{APPUNTI BART}
		
		\begin{thebibliography}{9}
			\bibitem{tf}
			Google Research,
			\emph{TensorFlow: Large-Scale Machine Learning on Heterogeneous Distributed Systems},
			Preliminary White Paper,
			2015.
		\end{thebibliography}
	\end{multicols}

\end{document}